\section{Introduction}
\label{sec:introduction}

Modern web services rely on complex authorization infrastructure that introduces operational overhead, security vulnerabilities, and vendor lock-in. A typical API service requires OAuth/OIDC servers for authentication, databases for session and subscription state management, payment processors for recurring billing, and credential management systems for revocation. This architecture creates single points of failure, requires continuous maintenance, and fragments audit trails across multiple systems. The security implications are severe: credential-based attacks now account for 38\% of all data breaches~\cite{verizon2024}, with organizations facing an average cost of \$4.88 million per breach~\cite{ibm2024breach}. Microsoft reports over 600 million identity attacks occurring daily~\cite{microsoft2024}, while authentication-as-a-service platforms cost \$500--\$2,000 per month for OAuth infrastructure alone~\cite{auth0pricing}. For example, revoking access requires coordinating updates across authorization servers, session stores, and client caches, with propagation delays that can leave systems vulnerable to unauthorized access.

The rise of AI agents and autonomous systems has exposed fundamental limitations in traditional authorization models. AI agents such as Claude, GPT-4, and autonomous MCP (Model Context Protocol) servers require programmatic API access without human intervention for interactive OAuth flows~\cite{oauth2}. Current solutions either resort to long-lived API keys stored in environment variables (which cannot be easily revoked or expire automatically) or require complex delegation mechanisms through OAuth 2.0 client credentials flow that reintroduce centralized control points. Emerging standards like ERC-8004 (Trustless Agents)~\cite{erc8004} and session key delegation (ERC-7710)~\cite{erc7710} attempt to address agent identity and capability delegation, but no existing solution provides native programmatic API authorization via token ownership without human-interactive consent flows. As AI agents increasingly mediate access to services on behalf of users, the need for decentralized, programmatic authorization becomes critical.

Blockchain technology offers an alternative paradigm: authorization as verifiable state rather than centralized credential management. Token ownership on a public blockchain provides persistent, cryptographically verifiable proof of access rights. The blockchain's immutable ledger creates a complete audit trail, while smart contracts enable atomic operations that combine payment and authorization in a single transaction. However, existing token standards (ERC-20~\cite{erc20}, ERC-721~\cite{erc721}, ERC-1155~\cite{erc1155}, ERC-6909~\cite{erc6909}) were designed for asset ownership rather than authorization, lacking features essential for access control: time-bounded validity (no native expiration mechanisms), non-transferability (all standards default to transferable tokens; soulbound extensions like EIP-5192~\cite{eip5192} remain non-standard), instant revocation (requiring transaction gas costs and lacking emergency pause mechanisms), and direct purchase mechanisms integrated with authorization logic. Similarly, existing blockchain-based access control solutions like Collab.Land and Guild.xyz rely on off-chain verification with update latencies of 24 hours to 7 days~\cite{collabland}, while Unlock Protocol~\cite{unlock} targets human-centric subscription models rather than programmatic API authorization.

\subsection{Contributions}

This paper presents EVMAuth, a comprehensive on-chain authorization framework that eliminates traditional authorization infrastructure dependencies. EVMAuth makes the following contributions:

\begin{itemize}
\item \textbf{Composable Authorization Primitives:} We introduce seven orthogonal features that compose to support diverse access control patterns: ephemeral tokens with automatic expiration, soulbound tokens that prevent transfers, account freezing for instant revocation, token enumeration for subscription discovery, transfer control for tiered access, direct purchase for atomic payment-authorization, and role-based access control for administrative operations.

\item \textbf{Dual-Standard Implementation:} We provide production-ready implementations based on both ERC-1155 (for batch operations and metadata flexibility) and ERC-6909 (for minimal gas overhead), demonstrating how authorization primitives can be integrated into different token architectures. Our benchmarks show deployment costs of 5.4M gas (ERC-1155) and 4.9M gas (ERC-6909), with operation costs ranging from 14K gas for configuration updates to 1.2M gas for pruning 100 expired records.

\item \textbf{Bounded-Storage Ephemeral Tokens:} Building on our time-bucketed balance records algorithm~\cite{paper1}, we demonstrate how bounded storage ($k+1$ records) enables predictable gas costs while supporting automatic expiration. With $k=100$ and a 30-day TTL, worst-case transfer operations require 11M gas, trading storage bounds for computational cost in a way that remains practical for authorization use cases.

\item \textbf{Real-World Deployment and Evaluation:} We deploy EVMAuth on Ethereum mainnet and evaluate it against traditional OAuth infrastructure across multiple dimensions: infrastructure cost, verification latency, revocation propagation, audit trail integrity, and operational complexity. Our comparison shows that EVMAuth eliminates monthly server costs (vs. \$500-2000 for OAuth infrastructure), achieves sub-100ms verification through cached RPC calls, and provides instant global revocation through on-chain state updates.
\end{itemize}

The remainder of this paper is organized as follows. Section~\ref{sec:background} provides background on traditional authorization architectures and blockchain token standards. Section~\ref{sec:design} presents the EVMAuth architecture and composable primitives. Section~\ref{sec:implementation} describes our dual-standard implementation and optimization techniques. Section~\ref{sec:evaluation} evaluates gas costs, compares with OAuth infrastructure, and analyzes the bounded storage trade-offs. Section~\ref{sec:discussion} discusses security considerations, limitations, and deployment experiences. Section~\ref{sec:related} surveys related work, and Section~\ref{sec:conclusion} concludes.
