\section{Algorithm Design}
\label{sec:algorithm}

\subsection{Time-Bucketing Mechanism}

The key insight is to discretize continuous expiration times into $k$ discrete bucket boundaries.
Given a TTL of $T$ seconds and a target of at most $k$ balance records per account, we define the bucket width:

\begin{equation}
w = \left\lceil \frac{T}{k} \right\rceil
\label{eq:bucket-width}
\end{equation}

For a deposit at time $t$, the exact expiration would be $t + T$.
We compute the \emph{bucketed expiration} by rounding up to the next bucket boundary:

\begin{equation}
\textsc{BucketedExpiry}(t, T, w) = \left\lceil \frac{t + T}{w} \right\rceil \times w
\label{eq:bucketed-expiry}
\end{equation}

The ceiling operation in Equation~\ref{eq:bucketed-expiry} \emph{never decreases} the expiration time, ensuring tokens never expire before their configured TTL (satisfying requirement R2 from Section~\ref{sec:problem}).
The additional lifetime granted by rounding is at most $w - 1$ seconds---a configurable trade-off between expiration precision and storage efficiency.
Figure~\ref{fig:bucketing} illustrates this mechanism.

% Time-Bucketing Diagram
% Include in main document with: % Time-Bucketing Diagram
% Include in main document with: % Time-Bucketing Diagram
% Include in main document with: \input{figures/time-bucketing}

\begin{figure}[t]
\centering
\begin{tikzpicture}[
    >=stealth,
    bucket/.style={draw, minimum height=0.6cm, minimum width=2.2cm, fill=gray!10},
    deposit/.style={circle, fill=black, inner sep=2pt},
    expiry/.style={circle, draw, inner sep=2pt},
    bucketed/.style={diamond, fill=black, inner sep=2pt},
    arrow/.style={->, thick},
    dashedarrow/.style={->, dashed, gray}
]

% === Part A: Bucket Structure ===
\node[anchor=west, font=\footnotesize\bfseries] at (-0.5, 3.8) {(a) Time divided into $k$ buckets};

% Timeline
\draw[thick, ->] (0, 2.5) -- (10, 2.5) node[right] {\footnotesize time};

% Bucket boxes
\foreach \i/\label in {0/1, 1/2, 2/3, 3/4} {
    \node[bucket] at (1.1 + \i*2.2, 2.5) {\footnotesize Bucket \label};
}

% Bucket boundaries
\foreach \i/\label in {0/$b_1$, 1/$b_2$, 2/$b_3$, 3/$b_4$, 4/$b_5$} {
    \draw[thick] (\i*2.2, 2.2) -- (\i*2.2, 2.8);
    \node[below, font=\footnotesize] at (\i*2.2, 2.15) {\label};
}

% TTL annotation
\draw[<->] (0, 3.0) -- (8.8, 3.0);
\node[above, font=\footnotesize] at (4.4, 3.0) {TTL = $T$};

% Bucket size annotation
\draw[<->] (0, 1.7) -- (2.2, 1.7);
\node[below, font=\footnotesize] at (1.1, 1.75) {$w = \lceil T/k \rceil$};

% === Part B: Deposit to Bucket Mapping ===
\node[anchor=west, font=\footnotesize\bfseries] at (-0.5, 0.9) {(b) Deposits mapped to bucket boundaries};

% Timeline
\draw[thick, ->] (0, 0) -- (10, 0) node[right] {\footnotesize time};

% Bucket boundaries (lighter)
\foreach \i in {0, 1, 2, 3, 4} {
    \draw[gray, dashed] (\i*2.2, -0.3) -- (\i*2.2, 0.3);
}

% Deposits
\node[deposit, label={above:\footnotesize $D_1$}] (d1) at (0.8, 0) {};
\node[deposit, label={above:\footnotesize $D_2$}] (d2) at (3.5, 0) {};
\node[deposit, label={above:\footnotesize $D_3$}] (d3) at (6.0, 0) {};

% Exact expiries (below timeline)
\node[expiry, label={below:\footnotesize exact}] (e1) at (3.0, -1.2) {};
\node[expiry, label={below:\footnotesize exact}] (e2) at (5.7, -1.2) {};
\node[expiry, label={below:\footnotesize exact}] (e3) at (8.2, -1.2) {};

% Arrows from deposits to exact expiries
\draw[dashedarrow] (d1) -- node[right, font=\tiny, pos=0.5] {$+T$} (e1);
\draw[dashedarrow] (d2) -- node[right, font=\tiny, pos=0.5] {$+T$} (e2);
\draw[dashedarrow] (d3) -- node[right, font=\tiny, pos=0.5] {$+T$} (e3);

% Bucketed expiries (on timeline markers)
\node[bucketed] (b1) at (4.4, -1.2) {};
\node[bucketed] (b2) at (6.6, -1.2) {};
\node[bucketed] (b3) at (8.8, -1.2) {};

% Arrows from exact to bucketed (ceiling)
\draw[arrow, blue] (e1) -- node[above, font=\tiny] {$\lceil\cdot\rceil$} (b1);
\draw[arrow, blue] (e2) -- node[above, font=\tiny] {$\lceil\cdot\rceil$} (b2);
\draw[arrow, blue] (e3) -- node[above, font=\tiny] {$\lceil\cdot\rceil$} (b3);

% Labels for bucketed expiries
\node[below, font=\footnotesize] at (4.4, -1.5) {$b_3$};
\node[below, font=\footnotesize] at (6.6, -1.5) {$b_4$};
\node[below, font=\footnotesize] at (8.8, -1.5) {$b_5$};

% === Part C: Resulting Records ===
\node[anchor=west, font=\footnotesize\bfseries] at (-0.5, -2.5) {(c) Balance records after coalescing};

% Table
\draw (0, -3.0) rectangle (4.5, -4.5);
\draw (0, -3.5) -- (4.5, -3.5);
\draw (2.0, -3.0) -- (2.0, -4.5);

% Headers
\node[font=\footnotesize\bfseries] at (1.0, -3.25) {amount};
\node[font=\footnotesize\bfseries] at (3.25, -3.25) {expiresAt};

% Row 1
\draw (0, -4.0) -- (4.5, -4.0);
\node[font=\footnotesize] at (1.0, -3.75) {$a_1 + a_2$};
\node[font=\footnotesize] at (3.25, -3.75) {$b_4$};

% Row 2
\node[font=\footnotesize] at (1.0, -4.25) {$a_3$};
\node[font=\footnotesize] at (3.25, -4.25) {$b_5$};

% % Annotation
\node[anchor=west, font=\scriptsize, text=blue] at (-0.2, -0.95) {$D_1, D_2$ coalesced};
\node[anchor=west, font=\scriptsize, gray] at (-0.2, -1.25) {(same bucket)};

% Legend
\node[anchor=west, font=\footnotesize] at (6.5, -2.8) {\textbf{Legend:}};
\node[deposit, label={right:\footnotesize deposit}] at (6.7, -3.2) {};
\node[expiry, label={right:\footnotesize exact expiry}] at (6.7, -3.6) {};
\node[bucketed, label={right:\footnotesize bucketed expiry}] at (6.7, -4.0) {};

\end{tikzpicture}
\caption{Time-bucketing mechanism. (a) The TTL period $T$ is divided into $k$ buckets of width $w = \lceil T/k \rceil$. (b) Each deposit $D_i$ is assigned a bucketed expiration by rounding $t_i + T$ up to the next bucket boundary via Equation~\ref{eq:bucketed-expiry}. (c) Deposits with identical bucketed expirations coalesce into a single balance record, bounding storage to at most $k$ records per account.}
\label{fig:bucketing}
\end{figure}


\begin{figure}[t]
\centering
\begin{tikzpicture}[
    >=stealth,
    bucket/.style={draw, minimum height=0.6cm, minimum width=2.2cm, fill=gray!10},
    deposit/.style={circle, fill=black, inner sep=2pt},
    expiry/.style={circle, draw, inner sep=2pt},
    bucketed/.style={diamond, fill=black, inner sep=2pt},
    arrow/.style={->, thick},
    dashedarrow/.style={->, dashed, gray}
]

% === Part A: Bucket Structure ===
\node[anchor=west, font=\footnotesize\bfseries] at (-0.5, 3.8) {(a) Time divided into $k$ buckets};

% Timeline
\draw[thick, ->] (0, 2.5) -- (10, 2.5) node[right] {\footnotesize time};

% Bucket boxes
\foreach \i/\label in {0/1, 1/2, 2/3, 3/4} {
    \node[bucket] at (1.1 + \i*2.2, 2.5) {\footnotesize Bucket \label};
}

% Bucket boundaries
\foreach \i/\label in {0/$b_1$, 1/$b_2$, 2/$b_3$, 3/$b_4$, 4/$b_5$} {
    \draw[thick] (\i*2.2, 2.2) -- (\i*2.2, 2.8);
    \node[below, font=\footnotesize] at (\i*2.2, 2.15) {\label};
}

% TTL annotation
\draw[<->] (0, 3.0) -- (8.8, 3.0);
\node[above, font=\footnotesize] at (4.4, 3.0) {TTL = $T$};

% Bucket size annotation
\draw[<->] (0, 1.7) -- (2.2, 1.7);
\node[below, font=\footnotesize] at (1.1, 1.75) {$w = \lceil T/k \rceil$};

% === Part B: Deposit to Bucket Mapping ===
\node[anchor=west, font=\footnotesize\bfseries] at (-0.5, 0.9) {(b) Deposits mapped to bucket boundaries};

% Timeline
\draw[thick, ->] (0, 0) -- (10, 0) node[right] {\footnotesize time};

% Bucket boundaries (lighter)
\foreach \i in {0, 1, 2, 3, 4} {
    \draw[gray, dashed] (\i*2.2, -0.3) -- (\i*2.2, 0.3);
}

% Deposits
\node[deposit, label={above:\footnotesize $D_1$}] (d1) at (0.8, 0) {};
\node[deposit, label={above:\footnotesize $D_2$}] (d2) at (3.5, 0) {};
\node[deposit, label={above:\footnotesize $D_3$}] (d3) at (6.0, 0) {};

% Exact expiries (below timeline)
\node[expiry, label={below:\footnotesize exact}] (e1) at (3.0, -1.2) {};
\node[expiry, label={below:\footnotesize exact}] (e2) at (5.7, -1.2) {};
\node[expiry, label={below:\footnotesize exact}] (e3) at (8.2, -1.2) {};

% Arrows from deposits to exact expiries
\draw[dashedarrow] (d1) -- node[right, font=\tiny, pos=0.5] {$+T$} (e1);
\draw[dashedarrow] (d2) -- node[right, font=\tiny, pos=0.5] {$+T$} (e2);
\draw[dashedarrow] (d3) -- node[right, font=\tiny, pos=0.5] {$+T$} (e3);

% Bucketed expiries (on timeline markers)
\node[bucketed] (b1) at (4.4, -1.2) {};
\node[bucketed] (b2) at (6.6, -1.2) {};
\node[bucketed] (b3) at (8.8, -1.2) {};

% Arrows from exact to bucketed (ceiling)
\draw[arrow, blue] (e1) -- node[above, font=\tiny] {$\lceil\cdot\rceil$} (b1);
\draw[arrow, blue] (e2) -- node[above, font=\tiny] {$\lceil\cdot\rceil$} (b2);
\draw[arrow, blue] (e3) -- node[above, font=\tiny] {$\lceil\cdot\rceil$} (b3);

% Labels for bucketed expiries
\node[below, font=\footnotesize] at (4.4, -1.5) {$b_3$};
\node[below, font=\footnotesize] at (6.6, -1.5) {$b_4$};
\node[below, font=\footnotesize] at (8.8, -1.5) {$b_5$};

% === Part C: Resulting Records ===
\node[anchor=west, font=\footnotesize\bfseries] at (-0.5, -2.5) {(c) Balance records after coalescing};

% Table
\draw (0, -3.0) rectangle (4.5, -4.5);
\draw (0, -3.5) -- (4.5, -3.5);
\draw (2.0, -3.0) -- (2.0, -4.5);

% Headers
\node[font=\footnotesize\bfseries] at (1.0, -3.25) {amount};
\node[font=\footnotesize\bfseries] at (3.25, -3.25) {expiresAt};

% Row 1
\draw (0, -4.0) -- (4.5, -4.0);
\node[font=\footnotesize] at (1.0, -3.75) {$a_1 + a_2$};
\node[font=\footnotesize] at (3.25, -3.75) {$b_4$};

% Row 2
\node[font=\footnotesize] at (1.0, -4.25) {$a_3$};
\node[font=\footnotesize] at (3.25, -4.25) {$b_5$};

% % Annotation
\node[anchor=west, font=\scriptsize, text=blue] at (-0.2, -0.95) {$D_1, D_2$ coalesced};
\node[anchor=west, font=\scriptsize, gray] at (-0.2, -1.25) {(same bucket)};

% Legend
\node[anchor=west, font=\footnotesize] at (6.5, -2.8) {\textbf{Legend:}};
\node[deposit, label={right:\footnotesize deposit}] at (6.7, -3.2) {};
\node[expiry, label={right:\footnotesize exact expiry}] at (6.7, -3.6) {};
\node[bucketed, label={right:\footnotesize bucketed expiry}] at (6.7, -4.0) {};

\end{tikzpicture}
\caption{Time-bucketing mechanism. (a) The TTL period $T$ is divided into $k$ buckets of width $w = \lceil T/k \rceil$. (b) Each deposit $D_i$ is assigned a bucketed expiration by rounding $t_i + T$ up to the next bucket boundary via Equation~\ref{eq:bucketed-expiry}. (c) Deposits with identical bucketed expirations coalesce into a single balance record, bounding storage to at most $k$ records per account.}
\label{fig:bucketing}
\end{figure}


\begin{figure}[t]
\centering
\begin{tikzpicture}[
    >=stealth,
    bucket/.style={draw, minimum height=0.6cm, minimum width=2.2cm, fill=gray!10},
    deposit/.style={circle, fill=black, inner sep=2pt},
    expiry/.style={circle, draw, inner sep=2pt},
    bucketed/.style={diamond, fill=black, inner sep=2pt},
    arrow/.style={->, thick},
    dashedarrow/.style={->, dashed, gray}
]

% === Part A: Bucket Structure ===
\node[anchor=west, font=\footnotesize\bfseries] at (-0.5, 3.8) {(a) Time divided into $k$ buckets};

% Timeline
\draw[thick, ->] (0, 2.5) -- (10, 2.5) node[right] {\footnotesize time};

% Bucket boxes
\foreach \i/\label in {0/1, 1/2, 2/3, 3/4} {
    \node[bucket] at (1.1 + \i*2.2, 2.5) {\footnotesize Bucket \label};
}

% Bucket boundaries
\foreach \i/\label in {0/$b_1$, 1/$b_2$, 2/$b_3$, 3/$b_4$, 4/$b_5$} {
    \draw[thick] (\i*2.2, 2.2) -- (\i*2.2, 2.8);
    \node[below, font=\footnotesize] at (\i*2.2, 2.15) {\label};
}

% TTL annotation
\draw[<->] (0, 3.0) -- (8.8, 3.0);
\node[above, font=\footnotesize] at (4.4, 3.0) {TTL = $T$};

% Bucket size annotation
\draw[<->] (0, 1.7) -- (2.2, 1.7);
\node[below, font=\footnotesize] at (1.1, 1.75) {$w = \lceil T/k \rceil$};

% === Part B: Deposit to Bucket Mapping ===
\node[anchor=west, font=\footnotesize\bfseries] at (-0.5, 0.9) {(b) Deposits mapped to bucket boundaries};

% Timeline
\draw[thick, ->] (0, 0) -- (10, 0) node[right] {\footnotesize time};

% Bucket boundaries (lighter)
\foreach \i in {0, 1, 2, 3, 4} {
    \draw[gray, dashed] (\i*2.2, -0.3) -- (\i*2.2, 0.3);
}

% Deposits
\node[deposit, label={above:\footnotesize $D_1$}] (d1) at (0.8, 0) {};
\node[deposit, label={above:\footnotesize $D_2$}] (d2) at (3.5, 0) {};
\node[deposit, label={above:\footnotesize $D_3$}] (d3) at (6.0, 0) {};

% Exact expiries (below timeline)
\node[expiry, label={below:\footnotesize exact}] (e1) at (3.0, -1.2) {};
\node[expiry, label={below:\footnotesize exact}] (e2) at (5.7, -1.2) {};
\node[expiry, label={below:\footnotesize exact}] (e3) at (8.2, -1.2) {};

% Arrows from deposits to exact expiries
\draw[dashedarrow] (d1) -- node[right, font=\tiny, pos=0.5] {$+T$} (e1);
\draw[dashedarrow] (d2) -- node[right, font=\tiny, pos=0.5] {$+T$} (e2);
\draw[dashedarrow] (d3) -- node[right, font=\tiny, pos=0.5] {$+T$} (e3);

% Bucketed expiries (on timeline markers)
\node[bucketed] (b1) at (4.4, -1.2) {};
\node[bucketed] (b2) at (6.6, -1.2) {};
\node[bucketed] (b3) at (8.8, -1.2) {};

% Arrows from exact to bucketed (ceiling)
\draw[arrow, blue] (e1) -- node[above, font=\tiny] {$\lceil\cdot\rceil$} (b1);
\draw[arrow, blue] (e2) -- node[above, font=\tiny] {$\lceil\cdot\rceil$} (b2);
\draw[arrow, blue] (e3) -- node[above, font=\tiny] {$\lceil\cdot\rceil$} (b3);

% Labels for bucketed expiries
\node[below, font=\footnotesize] at (4.4, -1.5) {$b_3$};
\node[below, font=\footnotesize] at (6.6, -1.5) {$b_4$};
\node[below, font=\footnotesize] at (8.8, -1.5) {$b_5$};

% === Part C: Resulting Records ===
\node[anchor=west, font=\footnotesize\bfseries] at (-0.5, -2.5) {(c) Balance records after coalescing};

% Table
\draw (0, -3.0) rectangle (4.5, -4.5);
\draw (0, -3.5) -- (4.5, -3.5);
\draw (2.0, -3.0) -- (2.0, -4.5);

% Headers
\node[font=\footnotesize\bfseries] at (1.0, -3.25) {amount};
\node[font=\footnotesize\bfseries] at (3.25, -3.25) {expiresAt};

% Row 1
\draw (0, -4.0) -- (4.5, -4.0);
\node[font=\footnotesize] at (1.0, -3.75) {$a_1 + a_2$};
\node[font=\footnotesize] at (3.25, -3.75) {$b_4$};

% Row 2
\node[font=\footnotesize] at (1.0, -4.25) {$a_3$};
\node[font=\footnotesize] at (3.25, -4.25) {$b_5$};

% % Annotation
\node[anchor=west, font=\scriptsize, text=blue] at (-0.2, -0.95) {$D_1, D_2$ coalesced};
\node[anchor=west, font=\scriptsize, gray] at (-0.2, -1.25) {(same bucket)};

% Legend
\node[anchor=west, font=\footnotesize] at (6.5, -2.8) {\textbf{Legend:}};
\node[deposit, label={right:\footnotesize deposit}] at (6.7, -3.2) {};
\node[expiry, label={right:\footnotesize exact expiry}] at (6.7, -3.6) {};
\node[bucketed, label={right:\footnotesize bucketed expiry}] at (6.7, -4.0) {};

\end{tikzpicture}
\caption{Time-bucketing mechanism. (a) The TTL period $T$ is divided into $k$ buckets of width $w = \lceil T/k \rceil$. (b) Each deposit $D_i$ is assigned a bucketed expiration by rounding $t_i + T$ up to the next bucket boundary via Equation~\ref{eq:bucketed-expiry}. (c) Deposits with identical bucketed expirations coalesce into a single balance record, bounding storage to at most $k$ records per account.}
\label{fig:bucketing}
\end{figure}


\subsection{Data Structure}

We maintain a sorted array of balance records per account:

\begin{equation}
\mathcal{B} = [(a_1, e_1), (a_2, e_2), \ldots, (a_n, e_n)]
\label{eq:balance-records}
\end{equation}

\noindent where $a_i > 0$ is the token amount and $e_i$ is the bucketed expiration timestamp, ordered such that $e_1 < e_2 < \ldots < e_n$.

\textbf{Invariant.} All expiration timestamps in $\mathcal{B}$ are \emph{distinct} bucket boundaries.
The active expiration window spans from the earliest non-expired bucket to the latest possible new deposit expiration, containing at most $k+1$ distinct bucket boundaries, ensuring $|\mathcal{B}| \leq k+1$ (satisfying requirement R1).

\subsection{Operations}

We define four core operations.
Algorithm~\ref{alg:operations} presents the pseudocode for \textsc{Insert}, \textsc{Consume}, \textsc{Transfer}, and \textsc{Prune}.

\begin{algorithm}[htbp]
\caption{Time-Bucketed Balance Record Operations}
\label{alg:operations}
\small
\begin{algorithmic}[1]
\Require Balance records $\mathcal{B}$, bucket width $w$, current time $t_{\text{now}}$

\Function{BucketedExpiry}{$t$, $T$, $w$}
    \State \Return $\lceil (t + T) / w \rceil \times w$
\EndFunction

\Function{Insert}{$\mathcal{B}$, $a$, $e$, $t_{\text{now}}$}
    \State \Call{Prune}{$\mathcal{B}$, $t_{\text{now}}$} \Comment{Remove expired records}
    \For{$i \gets 0$ \textbf{to} $|\mathcal{B}| - 1$}
        \If{$\mathcal{B}[i].e = e$}
            $\mathcal{B}[i].a \gets \mathcal{B}[i].a + a$; \Return \Comment{Coalesce}
        \EndIf
        \If{$\mathcal{B}[i].e > e$}
            \textsc{ShiftInsert}$(\mathcal{B}, i, (a, e))$; \Return
        \EndIf
    \EndFor
    \State $\mathcal{B}.\text{append}((a, e))$
\EndFunction

\Function{Consume}{$\mathcal{B}$, $a$, $t_{\text{now}}$}
    \State $r \gets a$; $C \gets []$ \Comment{Remaining; collected pairs}
    \For{$i \gets 0$ \textbf{to} $|\mathcal{B}| - 1$}
        \If{$\mathcal{B}[i].e \leq t_{\text{now}}$} \textbf{continue} \EndIf \Comment{Skip expired}
        \State $\delta \gets \min(r, \mathcal{B}[i].a)$
        \State $\mathcal{B}[i].a \gets \mathcal{B}[i].a - \delta$; $C.\text{append}((\delta, \mathcal{B}[i].e))$
        \State $r \gets r - \delta$
        \If{$r = 0$} \Return $(\textsc{Success}, C)$ \EndIf
    \EndFor
    \State \Return $(\textsc{InsufficientBalance}, \emptyset)$
\EndFunction

\Function{Transfer}{$\mathcal{B}_s$, $\mathcal{B}_r$, $a$, $t_{\text{now}}$}
    \State $(status, C) \gets$ \Call{Consume}{$\mathcal{B}_s$, $a$, $t_{\text{now}}$}
    \If{$status \neq \textsc{Success}$} \Return $status$ \EndIf
    \For{$(a_i, e_i) \in C$}
        \Call{Insert}{$\mathcal{B}_r$, $a_i$, $e_i$, $t_{\text{now}}$} \Comment{Preserve expiration}
    \EndFor
    \State \Return \textsc{Success}
\EndFunction

\Function{Prune}{$\mathcal{B}$, $t_{\text{now}}$}
    \State $j \gets 0$
    \For{$i \gets 0$ \textbf{to} $|\mathcal{B}| - 1$}
        \If{$\mathcal{B}[i].e > t_{\text{now}}$ \textbf{and} $\mathcal{B}[i].a > 0$}
            $\mathcal{B}[j] \gets \mathcal{B}[i]$; $j \gets j + 1$
        \EndIf
    \EndFor
    \State $\mathcal{B}.\text{truncate}(j)$
\EndFunction

\end{algorithmic}
\end{algorithm}


\paragraph{Insert.}
To deposit amount $a$ with bucketed expiration $e$:
(1) search for an existing record with expiration $e$;
(2) if found, add $a$ to the existing record's amount (\emph{coalescing});
(3) otherwise, insert a new record $(a, e)$ maintaining sorted order.
Coalescing is the key mechanism that bounds storage: multiple deposits mapping to the same bucket boundary share a single record.

\paragraph{Consume.}
To withdraw amount $a$ (for burns or the sender side of transfers):
(1) iterate through records from oldest (earliest expiration) to newest;
(2) skip expired records where $e_i \leq t_{\text{now}}$;
(3) deduct from each valid record until $a$ is fully satisfied.
This implements FIFO semantics (requirement R3): oldest tokens are consumed first.

\paragraph{Transfer.}
To transfer amount $a$ from sender to recipient:
(1) consume from sender using FIFO, collecting $(amount, expiration)$ pairs rather than discarding them;
(2) insert each collected pair into the recipient's records using the \emph{original} expiration timestamp.
Crucially, expirations are \emph{not} re-bucketed on transfer---the recipient inherits the sender's expiration times exactly, satisfying requirement R4 (expiration-preserving transfers).

\paragraph{Prune.}
To remove stale records:
(1) iterate through all records;
(2) compact valid records (non-expired with $e_i > t_{\text{now}}$ and non-zero $a_i > 0$) to the front of the array;
(3) truncate the array.
Pruning is invoked automatically before insertions to reclaim storage from expired records.
