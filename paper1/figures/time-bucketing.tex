% Time-Bucketing Diagram
% Include in main document with: % Time-Bucketing Diagram
% Include in main document with: % Time-Bucketing Diagram
% Include in main document with: % Time-Bucketing Diagram
% Include in main document with: \input{figures/time-bucketing}

\begin{figure}[t]
\centering
\begin{tikzpicture}[
    >=stealth,
    bucket/.style={draw, minimum height=0.6cm, minimum width=2.2cm, fill=gray!10},
    deposit/.style={circle, fill=black, inner sep=2pt},
    expiry/.style={circle, draw, inner sep=2pt},
    bucketed/.style={diamond, fill=black, inner sep=2pt},
    arrow/.style={->, thick},
    dashedarrow/.style={->, dashed, gray}
]

% === Part A: Bucket Structure ===
\node[anchor=west, font=\footnotesize\bfseries] at (-0.5, 3.8) {(a) Time divided into $k$ buckets};

% Timeline
\draw[thick, ->] (0, 2.5) -- (10, 2.5) node[right] {\footnotesize time};

% Bucket boxes
\foreach \i/\label in {0/1, 1/2, 2/3, 3/4} {
    \node[bucket] at (1.1 + \i*2.2, 2.5) {\footnotesize Bucket \label};
}

% Bucket boundaries
\foreach \i/\label in {0/$b_1$, 1/$b_2$, 2/$b_3$, 3/$b_4$, 4/$b_5$} {
    \draw[thick] (\i*2.2, 2.2) -- (\i*2.2, 2.8);
    \node[below, font=\footnotesize] at (\i*2.2, 2.15) {\label};
}

% TTL annotation
\draw[<->] (0, 3.0) -- (8.8, 3.0);
\node[above, font=\footnotesize] at (4.4, 3.0) {TTL = $T$};

% Bucket size annotation
\draw[<->] (0, 1.7) -- (2.2, 1.7);
\node[below, font=\footnotesize] at (1.1, 1.75) {$w = \lceil T/k \rceil$};

% === Part B: Deposit to Bucket Mapping ===
\node[anchor=west, font=\footnotesize\bfseries] at (-0.5, 0.9) {(b) Deposits mapped to bucket boundaries};

% Timeline
\draw[thick, ->] (0, 0) -- (10, 0) node[right] {\footnotesize time};

% Bucket boundaries (lighter)
\foreach \i in {0, 1, 2, 3, 4} {
    \draw[gray, dashed] (\i*2.2, -0.3) -- (\i*2.2, 0.3);
}

% Deposits
\node[deposit, label={above:\footnotesize $D_1$}] (d1) at (0.8, 0) {};
\node[deposit, label={above:\footnotesize $D_2$}] (d2) at (3.5, 0) {};
\node[deposit, label={above:\footnotesize $D_3$}] (d3) at (6.0, 0) {};

% Exact expiries (below timeline)
\node[expiry, label={below:\footnotesize exact}] (e1) at (3.0, -1.2) {};
\node[expiry, label={below:\footnotesize exact}] (e2) at (5.7, -1.2) {};
\node[expiry, label={below:\footnotesize exact}] (e3) at (8.2, -1.2) {};

% Arrows from deposits to exact expiries
\draw[dashedarrow] (d1) -- node[right, font=\tiny, pos=0.5] {$+T$} (e1);
\draw[dashedarrow] (d2) -- node[right, font=\tiny, pos=0.5] {$+T$} (e2);
\draw[dashedarrow] (d3) -- node[right, font=\tiny, pos=0.5] {$+T$} (e3);

% Bucketed expiries (on timeline markers)
\node[bucketed] (b1) at (4.4, -1.2) {};
\node[bucketed] (b2) at (6.6, -1.2) {};
\node[bucketed] (b3) at (8.8, -1.2) {};

% Arrows from exact to bucketed (ceiling)
\draw[arrow, blue] (e1) -- node[above, font=\tiny] {$\lceil\cdot\rceil$} (b1);
\draw[arrow, blue] (e2) -- node[above, font=\tiny] {$\lceil\cdot\rceil$} (b2);
\draw[arrow, blue] (e3) -- node[above, font=\tiny] {$\lceil\cdot\rceil$} (b3);

% Labels for bucketed expiries
\node[below, font=\footnotesize] at (4.4, -1.5) {$b_3$};
\node[below, font=\footnotesize] at (6.6, -1.5) {$b_4$};
\node[below, font=\footnotesize] at (8.8, -1.5) {$b_5$};

% === Part C: Resulting Records ===
\node[anchor=west, font=\footnotesize\bfseries] at (-0.5, -2.5) {(c) Balance records after coalescing};

% Table
\draw (0, -3.0) rectangle (4.5, -4.5);
\draw (0, -3.5) -- (4.5, -3.5);
\draw (2.0, -3.0) -- (2.0, -4.5);

% Headers
\node[font=\footnotesize\bfseries] at (1.0, -3.25) {amount};
\node[font=\footnotesize\bfseries] at (3.25, -3.25) {expiresAt};

% Row 1
\draw (0, -4.0) -- (4.5, -4.0);
\node[font=\footnotesize] at (1.0, -3.75) {$a_1 + a_2$};
\node[font=\footnotesize] at (3.25, -3.75) {$b_4$};

% Row 2
\node[font=\footnotesize] at (1.0, -4.25) {$a_3$};
\node[font=\footnotesize] at (3.25, -4.25) {$b_5$};

% % Annotation
\node[anchor=west, font=\scriptsize, text=blue] at (-0.2, -0.95) {$D_1, D_2$ coalesced};
\node[anchor=west, font=\scriptsize, gray] at (-0.2, -1.25) {(same bucket)};

% Legend
\node[anchor=west, font=\footnotesize] at (6.5, -2.8) {\textbf{Legend:}};
\node[deposit, label={right:\footnotesize deposit}] at (6.7, -3.2) {};
\node[expiry, label={right:\footnotesize exact expiry}] at (6.7, -3.6) {};
\node[bucketed, label={right:\footnotesize bucketed expiry}] at (6.7, -4.0) {};

\end{tikzpicture}
\caption{Time-bucketing mechanism. (a) The TTL period $T$ is divided into $k$ buckets of width $w = \lceil T/k \rceil$. (b) Each deposit $D_i$ is assigned a bucketed expiration by rounding $t_i + T$ up to the next bucket boundary via Equation~\ref{eq:bucketed-expiry}. (c) Deposits with identical bucketed expirations coalesce into a single balance record, bounding storage to at most $k$ records per account.}
\label{fig:bucketing}
\end{figure}


\begin{figure}[t]
\centering
\begin{tikzpicture}[
    >=stealth,
    bucket/.style={draw, minimum height=0.6cm, minimum width=2.2cm, fill=gray!10},
    deposit/.style={circle, fill=black, inner sep=2pt},
    expiry/.style={circle, draw, inner sep=2pt},
    bucketed/.style={diamond, fill=black, inner sep=2pt},
    arrow/.style={->, thick},
    dashedarrow/.style={->, dashed, gray}
]

% === Part A: Bucket Structure ===
\node[anchor=west, font=\footnotesize\bfseries] at (-0.5, 3.8) {(a) Time divided into $k$ buckets};

% Timeline
\draw[thick, ->] (0, 2.5) -- (10, 2.5) node[right] {\footnotesize time};

% Bucket boxes
\foreach \i/\label in {0/1, 1/2, 2/3, 3/4} {
    \node[bucket] at (1.1 + \i*2.2, 2.5) {\footnotesize Bucket \label};
}

% Bucket boundaries
\foreach \i/\label in {0/$b_1$, 1/$b_2$, 2/$b_3$, 3/$b_4$, 4/$b_5$} {
    \draw[thick] (\i*2.2, 2.2) -- (\i*2.2, 2.8);
    \node[below, font=\footnotesize] at (\i*2.2, 2.15) {\label};
}

% TTL annotation
\draw[<->] (0, 3.0) -- (8.8, 3.0);
\node[above, font=\footnotesize] at (4.4, 3.0) {TTL = $T$};

% Bucket size annotation
\draw[<->] (0, 1.7) -- (2.2, 1.7);
\node[below, font=\footnotesize] at (1.1, 1.75) {$w = \lceil T/k \rceil$};

% === Part B: Deposit to Bucket Mapping ===
\node[anchor=west, font=\footnotesize\bfseries] at (-0.5, 0.9) {(b) Deposits mapped to bucket boundaries};

% Timeline
\draw[thick, ->] (0, 0) -- (10, 0) node[right] {\footnotesize time};

% Bucket boundaries (lighter)
\foreach \i in {0, 1, 2, 3, 4} {
    \draw[gray, dashed] (\i*2.2, -0.3) -- (\i*2.2, 0.3);
}

% Deposits
\node[deposit, label={above:\footnotesize $D_1$}] (d1) at (0.8, 0) {};
\node[deposit, label={above:\footnotesize $D_2$}] (d2) at (3.5, 0) {};
\node[deposit, label={above:\footnotesize $D_3$}] (d3) at (6.0, 0) {};

% Exact expiries (below timeline)
\node[expiry, label={below:\footnotesize exact}] (e1) at (3.0, -1.2) {};
\node[expiry, label={below:\footnotesize exact}] (e2) at (5.7, -1.2) {};
\node[expiry, label={below:\footnotesize exact}] (e3) at (8.2, -1.2) {};

% Arrows from deposits to exact expiries
\draw[dashedarrow] (d1) -- node[right, font=\tiny, pos=0.5] {$+T$} (e1);
\draw[dashedarrow] (d2) -- node[right, font=\tiny, pos=0.5] {$+T$} (e2);
\draw[dashedarrow] (d3) -- node[right, font=\tiny, pos=0.5] {$+T$} (e3);

% Bucketed expiries (on timeline markers)
\node[bucketed] (b1) at (4.4, -1.2) {};
\node[bucketed] (b2) at (6.6, -1.2) {};
\node[bucketed] (b3) at (8.8, -1.2) {};

% Arrows from exact to bucketed (ceiling)
\draw[arrow, blue] (e1) -- node[above, font=\tiny] {$\lceil\cdot\rceil$} (b1);
\draw[arrow, blue] (e2) -- node[above, font=\tiny] {$\lceil\cdot\rceil$} (b2);
\draw[arrow, blue] (e3) -- node[above, font=\tiny] {$\lceil\cdot\rceil$} (b3);

% Labels for bucketed expiries
\node[below, font=\footnotesize] at (4.4, -1.5) {$b_3$};
\node[below, font=\footnotesize] at (6.6, -1.5) {$b_4$};
\node[below, font=\footnotesize] at (8.8, -1.5) {$b_5$};

% === Part C: Resulting Records ===
\node[anchor=west, font=\footnotesize\bfseries] at (-0.5, -2.5) {(c) Balance records after coalescing};

% Table
\draw (0, -3.0) rectangle (4.5, -4.5);
\draw (0, -3.5) -- (4.5, -3.5);
\draw (2.0, -3.0) -- (2.0, -4.5);

% Headers
\node[font=\footnotesize\bfseries] at (1.0, -3.25) {amount};
\node[font=\footnotesize\bfseries] at (3.25, -3.25) {expiresAt};

% Row 1
\draw (0, -4.0) -- (4.5, -4.0);
\node[font=\footnotesize] at (1.0, -3.75) {$a_1 + a_2$};
\node[font=\footnotesize] at (3.25, -3.75) {$b_4$};

% Row 2
\node[font=\footnotesize] at (1.0, -4.25) {$a_3$};
\node[font=\footnotesize] at (3.25, -4.25) {$b_5$};

% % Annotation
\node[anchor=west, font=\scriptsize, text=blue] at (-0.2, -0.95) {$D_1, D_2$ coalesced};
\node[anchor=west, font=\scriptsize, gray] at (-0.2, -1.25) {(same bucket)};

% Legend
\node[anchor=west, font=\footnotesize] at (6.5, -2.8) {\textbf{Legend:}};
\node[deposit, label={right:\footnotesize deposit}] at (6.7, -3.2) {};
\node[expiry, label={right:\footnotesize exact expiry}] at (6.7, -3.6) {};
\node[bucketed, label={right:\footnotesize bucketed expiry}] at (6.7, -4.0) {};

\end{tikzpicture}
\caption{Time-bucketing mechanism. (a) The TTL period $T$ is divided into $k$ buckets of width $w = \lceil T/k \rceil$. (b) Each deposit $D_i$ is assigned a bucketed expiration by rounding $t_i + T$ up to the next bucket boundary via Equation~\ref{eq:bucketed-expiry}. (c) Deposits with identical bucketed expirations coalesce into a single balance record, bounding storage to at most $k$ records per account.}
\label{fig:bucketing}
\end{figure}


\begin{figure}[t]
\centering
\begin{tikzpicture}[
    >=stealth,
    bucket/.style={draw, minimum height=0.6cm, minimum width=2.2cm, fill=gray!10},
    deposit/.style={circle, fill=black, inner sep=2pt},
    expiry/.style={circle, draw, inner sep=2pt},
    bucketed/.style={diamond, fill=black, inner sep=2pt},
    arrow/.style={->, thick},
    dashedarrow/.style={->, dashed, gray}
]

% === Part A: Bucket Structure ===
\node[anchor=west, font=\footnotesize\bfseries] at (-0.5, 3.8) {(a) Time divided into $k$ buckets};

% Timeline
\draw[thick, ->] (0, 2.5) -- (10, 2.5) node[right] {\footnotesize time};

% Bucket boxes
\foreach \i/\label in {0/1, 1/2, 2/3, 3/4} {
    \node[bucket] at (1.1 + \i*2.2, 2.5) {\footnotesize Bucket \label};
}

% Bucket boundaries
\foreach \i/\label in {0/$b_1$, 1/$b_2$, 2/$b_3$, 3/$b_4$, 4/$b_5$} {
    \draw[thick] (\i*2.2, 2.2) -- (\i*2.2, 2.8);
    \node[below, font=\footnotesize] at (\i*2.2, 2.15) {\label};
}

% TTL annotation
\draw[<->] (0, 3.0) -- (8.8, 3.0);
\node[above, font=\footnotesize] at (4.4, 3.0) {TTL = $T$};

% Bucket size annotation
\draw[<->] (0, 1.7) -- (2.2, 1.7);
\node[below, font=\footnotesize] at (1.1, 1.75) {$w = \lceil T/k \rceil$};

% === Part B: Deposit to Bucket Mapping ===
\node[anchor=west, font=\footnotesize\bfseries] at (-0.5, 0.9) {(b) Deposits mapped to bucket boundaries};

% Timeline
\draw[thick, ->] (0, 0) -- (10, 0) node[right] {\footnotesize time};

% Bucket boundaries (lighter)
\foreach \i in {0, 1, 2, 3, 4} {
    \draw[gray, dashed] (\i*2.2, -0.3) -- (\i*2.2, 0.3);
}

% Deposits
\node[deposit, label={above:\footnotesize $D_1$}] (d1) at (0.8, 0) {};
\node[deposit, label={above:\footnotesize $D_2$}] (d2) at (3.5, 0) {};
\node[deposit, label={above:\footnotesize $D_3$}] (d3) at (6.0, 0) {};

% Exact expiries (below timeline)
\node[expiry, label={below:\footnotesize exact}] (e1) at (3.0, -1.2) {};
\node[expiry, label={below:\footnotesize exact}] (e2) at (5.7, -1.2) {};
\node[expiry, label={below:\footnotesize exact}] (e3) at (8.2, -1.2) {};

% Arrows from deposits to exact expiries
\draw[dashedarrow] (d1) -- node[right, font=\tiny, pos=0.5] {$+T$} (e1);
\draw[dashedarrow] (d2) -- node[right, font=\tiny, pos=0.5] {$+T$} (e2);
\draw[dashedarrow] (d3) -- node[right, font=\tiny, pos=0.5] {$+T$} (e3);

% Bucketed expiries (on timeline markers)
\node[bucketed] (b1) at (4.4, -1.2) {};
\node[bucketed] (b2) at (6.6, -1.2) {};
\node[bucketed] (b3) at (8.8, -1.2) {};

% Arrows from exact to bucketed (ceiling)
\draw[arrow, blue] (e1) -- node[above, font=\tiny] {$\lceil\cdot\rceil$} (b1);
\draw[arrow, blue] (e2) -- node[above, font=\tiny] {$\lceil\cdot\rceil$} (b2);
\draw[arrow, blue] (e3) -- node[above, font=\tiny] {$\lceil\cdot\rceil$} (b3);

% Labels for bucketed expiries
\node[below, font=\footnotesize] at (4.4, -1.5) {$b_3$};
\node[below, font=\footnotesize] at (6.6, -1.5) {$b_4$};
\node[below, font=\footnotesize] at (8.8, -1.5) {$b_5$};

% === Part C: Resulting Records ===
\node[anchor=west, font=\footnotesize\bfseries] at (-0.5, -2.5) {(c) Balance records after coalescing};

% Table
\draw (0, -3.0) rectangle (4.5, -4.5);
\draw (0, -3.5) -- (4.5, -3.5);
\draw (2.0, -3.0) -- (2.0, -4.5);

% Headers
\node[font=\footnotesize\bfseries] at (1.0, -3.25) {amount};
\node[font=\footnotesize\bfseries] at (3.25, -3.25) {expiresAt};

% Row 1
\draw (0, -4.0) -- (4.5, -4.0);
\node[font=\footnotesize] at (1.0, -3.75) {$a_1 + a_2$};
\node[font=\footnotesize] at (3.25, -3.75) {$b_4$};

% Row 2
\node[font=\footnotesize] at (1.0, -4.25) {$a_3$};
\node[font=\footnotesize] at (3.25, -4.25) {$b_5$};

% % Annotation
\node[anchor=west, font=\scriptsize, text=blue] at (-0.2, -0.95) {$D_1, D_2$ coalesced};
\node[anchor=west, font=\scriptsize, gray] at (-0.2, -1.25) {(same bucket)};

% Legend
\node[anchor=west, font=\footnotesize] at (6.5, -2.8) {\textbf{Legend:}};
\node[deposit, label={right:\footnotesize deposit}] at (6.7, -3.2) {};
\node[expiry, label={right:\footnotesize exact expiry}] at (6.7, -3.6) {};
\node[bucketed, label={right:\footnotesize bucketed expiry}] at (6.7, -4.0) {};

\end{tikzpicture}
\caption{Time-bucketing mechanism. (a) The TTL period $T$ is divided into $k$ buckets of width $w = \lceil T/k \rceil$. (b) Each deposit $D_i$ is assigned a bucketed expiration by rounding $t_i + T$ up to the next bucket boundary via Equation~\ref{eq:bucketed-expiry}. (c) Deposits with identical bucketed expirations coalesce into a single balance record, bounding storage to at most $k$ records per account.}
\label{fig:bucketing}
\end{figure}


\begin{figure}[t]
\centering
\begin{tikzpicture}[
    >=stealth,
    bucket/.style={draw, minimum height=0.6cm, minimum width=2.2cm, fill=gray!10},
    deposit/.style={circle, fill=black, inner sep=2pt},
    expiry/.style={circle, draw, inner sep=2pt},
    bucketed/.style={diamond, fill=black, inner sep=2pt},
    arrow/.style={->, thick},
    dashedarrow/.style={->, dashed, gray}
]

% === Part A: Bucket Structure ===
\node[anchor=west, font=\footnotesize\bfseries] at (-0.5, 3.8) {(a) Time divided into $k$ buckets};

% Timeline
\draw[thick, ->] (0, 2.5) -- (10, 2.5) node[right] {\footnotesize time};

% Bucket boxes
\foreach \i/\label in {0/1, 1/2, 2/3, 3/4} {
    \node[bucket] at (1.1 + \i*2.2, 2.5) {\footnotesize Bucket \label};
}

% Bucket boundaries
\foreach \i/\label in {0/$b_1$, 1/$b_2$, 2/$b_3$, 3/$b_4$, 4/$b_5$} {
    \draw[thick] (\i*2.2, 2.2) -- (\i*2.2, 2.8);
    \node[below, font=\footnotesize] at (\i*2.2, 2.15) {\label};
}

% TTL annotation
\draw[<->] (0, 3.0) -- (8.8, 3.0);
\node[above, font=\footnotesize] at (4.4, 3.0) {TTL = $T$};

% Bucket size annotation
\draw[<->] (0, 1.7) -- (2.2, 1.7);
\node[below, font=\footnotesize] at (1.1, 1.75) {$w = \lceil T/k \rceil$};

% === Part B: Deposit to Bucket Mapping ===
\node[anchor=west, font=\footnotesize\bfseries] at (-0.5, 0.9) {(b) Deposits mapped to bucket boundaries};

% Timeline
\draw[thick, ->] (0, 0) -- (10, 0) node[right] {\footnotesize time};

% Bucket boundaries (lighter)
\foreach \i in {0, 1, 2, 3, 4} {
    \draw[gray, dashed] (\i*2.2, -0.3) -- (\i*2.2, 0.3);
}

% Deposits
\node[deposit, label={above:\footnotesize $D_1$}] (d1) at (0.8, 0) {};
\node[deposit, label={above:\footnotesize $D_2$}] (d2) at (3.5, 0) {};
\node[deposit, label={above:\footnotesize $D_3$}] (d3) at (6.0, 0) {};

% Exact expiries (below timeline)
\node[expiry, label={below:\footnotesize exact}] (e1) at (3.0, -1.2) {};
\node[expiry, label={below:\footnotesize exact}] (e2) at (5.7, -1.2) {};
\node[expiry, label={below:\footnotesize exact}] (e3) at (8.2, -1.2) {};

% Arrows from deposits to exact expiries
\draw[dashedarrow] (d1) -- node[right, font=\tiny, pos=0.5] {$+T$} (e1);
\draw[dashedarrow] (d2) -- node[right, font=\tiny, pos=0.5] {$+T$} (e2);
\draw[dashedarrow] (d3) -- node[right, font=\tiny, pos=0.5] {$+T$} (e3);

% Bucketed expiries (on timeline markers)
\node[bucketed] (b1) at (4.4, -1.2) {};
\node[bucketed] (b2) at (6.6, -1.2) {};
\node[bucketed] (b3) at (8.8, -1.2) {};

% Arrows from exact to bucketed (ceiling)
\draw[arrow, blue] (e1) -- node[above, font=\tiny] {$\lceil\cdot\rceil$} (b1);
\draw[arrow, blue] (e2) -- node[above, font=\tiny] {$\lceil\cdot\rceil$} (b2);
\draw[arrow, blue] (e3) -- node[above, font=\tiny] {$\lceil\cdot\rceil$} (b3);

% Labels for bucketed expiries
\node[below, font=\footnotesize] at (4.4, -1.5) {$b_3$};
\node[below, font=\footnotesize] at (6.6, -1.5) {$b_4$};
\node[below, font=\footnotesize] at (8.8, -1.5) {$b_5$};

% === Part C: Resulting Records ===
\node[anchor=west, font=\footnotesize\bfseries] at (-0.5, -2.5) {(c) Balance records after coalescing};

% Table
\draw (0, -3.0) rectangle (4.5, -4.5);
\draw (0, -3.5) -- (4.5, -3.5);
\draw (2.0, -3.0) -- (2.0, -4.5);

% Headers
\node[font=\footnotesize\bfseries] at (1.0, -3.25) {amount};
\node[font=\footnotesize\bfseries] at (3.25, -3.25) {expiresAt};

% Row 1
\draw (0, -4.0) -- (4.5, -4.0);
\node[font=\footnotesize] at (1.0, -3.75) {$a_1 + a_2$};
\node[font=\footnotesize] at (3.25, -3.75) {$b_4$};

% Row 2
\node[font=\footnotesize] at (1.0, -4.25) {$a_3$};
\node[font=\footnotesize] at (3.25, -4.25) {$b_5$};

% % Annotation
\node[anchor=west, font=\scriptsize, text=blue] at (-0.2, -0.95) {$D_1, D_2$ coalesced};
\node[anchor=west, font=\scriptsize, gray] at (-0.2, -1.25) {(same bucket)};

% Legend
\node[anchor=west, font=\footnotesize] at (6.5, -2.8) {\textbf{Legend:}};
\node[deposit, label={right:\footnotesize deposit}] at (6.7, -3.2) {};
\node[expiry, label={right:\footnotesize exact expiry}] at (6.7, -3.6) {};
\node[bucketed, label={right:\footnotesize bucketed expiry}] at (6.7, -4.0) {};

\end{tikzpicture}
\caption{Time-bucketing mechanism. (a) The TTL period $T$ is divided into $k$ buckets of width $w = \lceil T/k \rceil$. (b) Each deposit $D_i$ is assigned a bucketed expiration by rounding $t_i + T$ up to the next bucket boundary via Equation~\ref{eq:bucketed-expiry}. (c) Deposits with identical bucketed expirations coalesce into a single balance record, bounding storage to at most $k$ records per account.}
\label{fig:bucketing}
\end{figure}
